\documentclass{scrartcl}
\usepackage{todonotes}
\usepackage{multirow}
\usepackage{hyperref}
\usepackage{url}

\hypersetup{ colorlinks=true,
             pdftitle={HDF5 external filters - technical issues},
             citecolor=blue,
             urlcolor=blue,
             pagebackref=true
}

\title{
       {\huge HDF5 external filters}\\
       technical issues}
       
       \author{Eugen Wintersberger}

\begin{document}
\maketitle
\begin{abstract}
    This document tries to address all the technical issues concerning 
    the development and maintenance of external filter modules for HDF5. 
\end{abstract}

%%%===========================================================================
\section{Supported platforms and architectures}

The current release of HDF5\footnote{At the time of writting this was 1.8.13} 
supports the following platforms and architectures
\begin{center}
\begin{tabular}{l|c|c|c|c|c}
    & i386 & x86\_64 & SPARC 32-bit & SPARC 64-bit & PPC64 \\
    \hline\hline
    Linux & X  &  X & -- & -- & X \\
    \hline
    Windows 7 & X  & X & -- & -- & -- \\
    \hline
    Windows 8.1 & X & X & -- & -- & -- \\
    \hline
    OS X Lion 10.7.3 & -- & X & -- & -- & -- \\
    \hline
    OS X Mt. Lion 10.8.5 & -- & X & -- & -- & -- \\
    \hline
    OS X Mavericks 10.9.2 & -- & X & -- & -- & -- \\
    \hline
    SunOS 5.11 & -- & -- & X & X & -- \\
    \hline
    FreeBSD 8.2 & X & X & -- & -- & -- \\
    \hline
\end{tabular}
\end{center}
All external filter modules (EFMs) have to support the same set of platforms and 
architectures as the actual HDF5 release. This should ensure that no user 
experiences problems whenever he or she uses HDF5 on an officially supported
platform-architecture setup.

%%%===========================================================================
\section{Source code management}

Reasonable maintenance of software is only possible by using a source code
management (SCM) tool. The HDF5 library is currently using SVN\cite{SVN}
( as SCM system. However, for EF module
development SVN, being a centralized SCM tool, might not be the best choice. 
Due to its central nature it has some natural drawbacks
\begin{itemize}
    \item working offline is rather difficult 
    \item branching is as hard 
    \item it is hard to move repositories between different servers. 
\end{itemize}
In order to make things easy I suggest git~\cite{GIT} as the preferred SCM for EFM 
development. One of the big advantages of git is that repositories can be 
moved easily. This makes changing the hosting platform for an EFM project
simple. Such a process may becomes necessary if the original platform ceases 
to exist or if an abandoned module is taken over by a new developer who 
wants to use a different platform. 


%%%===========================================================================
\section{Compilers and code}

The choice of the programming language is rather simple: as the HDF5 core
library is written in C, C should also be used for the EFM. 
The language standard used is C99.

\subsection{Supported compilers and language standards}

All modules must comply to the C99 standard and build the code with the
following compilers 
\begin{center}
\begin{tabular}{l|| c | c}
    \hline
    \multirow{3}{*}{Windows} & intel cc & ?? \\
                             & mingw gcc  & ?? \\
     & MS Studio & ?? \\
    \hline
    \multirow{3}{*}{Linux} & gcc & ?? \\
     & intel & ??  \\
     & clang & ?? \\
    \hline
    OSX     & clang & ?? \\
    \hline
\end{tabular}
\todo{Check compilers and required versions}
\todo{is there a good reason why HDF5
does not support the MinGW compiler suite?}
\end{center}

\subsection{Code organization and coding style}

The code should be organized according to the GNU standards
\cite{GNU_STYLE,GNU_MAINTENANCE}.

\subsection{Build systems}

Due to the large number of platforms every EF module has to support at least two
build systems
\begin{description}
    \item[cmake] \cite{CMAKE} for builds on Windows and OSX
    \item[autotools] \cite{AUTOTOOLS} for builds on Linux and Unix systems.
\end{description}

\bibliographystyle{IEEEtran}
\bibliography{references}


\end{document}
